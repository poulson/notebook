\documentclass[letterpaper]{article}

\usepackage{hyperref}
\usepackage{geometry}

% Comment the following lines to use the default Computer Modern font
% instead of the Palatino font provided by the mathpazo package.
% Remove the 'osf' bit if you don't like the old style figures.
\usepackage[T1]{fontenc}
\usepackage[sc,osf]{mathpazo}

\def\name{Jack Poulson}

% Replace this with a link to your CV if you like, or set it empty
% (as in \def\footerlink{}) to remove the link in the footer:
\def\footerlink{}

% The following metadata will show up in the PDF properties
\hypersetup{
  colorlinks = true,
  urlcolor = black,
  pdfauthor = {\name},
  pdfkeywords = {linear algebra, parallel computing, mathematics},
  pdftitle = {\name: Curriculum Vitae},
  pdfsubject = {Curriculum Vitae},
  pdfpagemode = UseNone
}

\geometry{
  body={6.5in, 8.5in},
  left=1.0in,
  top=1.25in
}

% Customize page headers
\pagestyle{myheadings}
\markright{\name}
\thispagestyle{empty}

% Custom section fonts
\usepackage{sectsty}
\sectionfont{\rmfamily\mdseries\Large}
\subsectionfont{\rmfamily\mdseries\itshape\large}

% Don't indent paragraphs.
\setlength\parindent{0em}

% Make lists without bullets
\renewenvironment{itemize}{
  \begin{list}{}{
    \setlength{\leftmargin}{1.5em}
  }
}{
  \end{list}
}

\begin{document}

% Place name at left
{\huge \name}

\vspace{0.25in}

\begin{minipage}{0.45\linewidth}
  Department of Mathematics \\
  450 Serra Mall, Building 380 \\
  Stanford University \\
  Stanford, CA 94305-2125
\end{minipage}
\begin{minipage}{0.45\linewidth}
  \begin{tabular}{ll}
    Phone:&(404)431-9261 \\
    Email:&\href{mailto:poulson@stanford.edu}{\tt poulson@stanford.edu} \\
    Homepage: & \href{http://web.stanford.edu/~poulson}{\tt web.stanford.edu/\textasciitilde poulson} \\
  \end{tabular}
\end{minipage}

\section*{Research Objectives}
My research is increasingly focused on the development of efficient, 
open-source, distributed-memory libraries for dense and sparse-direct 
linear algebra and optimization, with a focus on scalable Interior Point 
Methods for conic programs. One of my long-term focuses is to provide scalable 
solvers for the entire conic and mixed-integer optimization hierarchy. 

\section*{Academic History}

\begin{tabular}{lll}
01/2015-        & {\bf Assistant Professor of Mathematics}               & Stanford University \\
10/2014-        & {\bf Member of Institute for Comput.\ Math.\ and Eng.} & Stanford University \\
10/2014-01/2015 & {\bf Acting Assistant Professor of Mathematics}        & Stanford University \\
11/2013-09/2014 & {\bf Assistant Professor of Comput.\ Science \& Eng. } & Georgia Institute of Technology \\
01/2013-11/2013 & {\bf Postdoctoral Scholar of Mathematics}              & Stanford University \\
12/2012         & {\bf PhD in Computational and Applied Mathematics}     & University of Texas at Austin \\
08/2011         & {\bf MS in Computational and Applied Mathematics}      & University of Texas at Austin \\
05/2009         & {\bf MS in Aerospace Engineering}                      & University of Texas at Austin \\
05/2007         & {\bf BS in Aerospace Engineering}                      & University of Texas at Austin \\
\end{tabular}

\section*{Industry and Laboratory Experience}

\begin{tabular}{llll}
06/2012-08/2012 & {\bf Research Intern} & WesternGeco                 & Distributed Lanczos for tomography \\
07/2010-08/2010 & {\bf Research Aide}   & Argonne National Laboratory & Eigensolvers for Blue Gene/P \\
06/2009-08/2009 & {\bf Research Intern} & Microsoft Research          & GPU-accelerated linear algebra \\
\end{tabular}

\section*{Academic Artifacts}

\subsection*{Parallel Mathematical Software}

\begin{itemize}
\item {\bf Elemental}, Primary Author, 2009-\\
A high-performance framework for distributed-memory dense and sparse-direct linear algebra and optimization. The functionality spans distributed sparse Interior Point Methods for Second-Order Cone Programs, high-performance distributed 
pseudospectra, and distributed sparse generalized Least Squares problems. \\ 
{\it Actively maintained and supported}\\
Available at 
\href{http://libelemental.org}{\tt libelemental.org} under the New BSD License.

\item {\bf Real-Time Low-rank Plus Sparse MRI}, Primary Author, 2013-\\
Distributed L+S decomposition for accelerated dynamic MRI.\\ 
{\it Prototype implementation}\\
Available at
\href{https://github.com/poulson/rt-lps-mri}{\tt github.com/poulson/rt-lps-mri} 
under the New BSD License.

\item {\bf DistButterfly}, Primary Author, 2010-\\
Distributed butterfly algorithms for non-uniform oscillatory transforms.\\
{\it Supported but not currently actively developed}\\
Available at
\href{https://github.com/poulson/dist-butterfly}{\tt github.com/poulson/dist-butterfly}
 under GPLv3.

\item {\bf Parallel Sweeping Preconditioner (PSP)}, Primary Author, 2011-\\
A prototype sweeping preconditioner for large 3d Helmholtz problems with  PML.\\
{\it Previously actively supported; should be reimplemented using Elemental}\\
Available at
\href{https://github.com/poulson/PSP}{\tt github.com/poulson/PSP} under GPLv3.

\item {\bf Distributed-Memory Hierarchical Matrices (DMHM)}, Primary Author, 2011-\\
A prototype of distributed-memory $\mathcal{H}$-matrix arithmetic. \\
{\it Proof of concept}\\
Available at
\href{https://bitbucket.org/poulson/dmhm}{\tt bitbucket.org/poulson/dmhm} under GPLv3.

%\item {\bf Elemental Krylov Spectral (EKS)}, Primary Author, Summer 2012 \\
%An implementation of the Krylov-spectral method geared towards large numbers of 
%eigenpairs. \\
%{\it Property of WesternGeco}

\end{itemize}

\subsection*{Peer-reviewed Journal Articles}

\begin{itemize}

\item Austin R. Benson, Jack Poulson, Kenneth Tran, Bj\"orn Engquist, and Lexing Ying.
A parallel directional Fast Multipole Method. {\it SIAM Journal on Scientific Computing},
36(4), C335--C352, 2014

\item Jack Poulson, Laurent Demanet, Nicholas Maxwell, and Lexing Ying.
A parallel butterfly algorithm, {\it SIAM Journal on Scientific Computing},
36(1), C49--C65, 2014

\item Paul Tsuji, Jack Poulson, Bj\"orn Engquist, and Lexing Ying.
Sweeping preconditioners for elastic wave propagation with spectral element
methods, {\it ESAIM: Mathematical Modeling and Numerical Analysis},
48(2), 433--447, 2014

\item Jack Poulson, Bj\"orn Engquist, Siwei Li, and Lexing Ying.
A parallel sweeping preconditioner for heterogeneous 3D Helmholtz equations, 
{\it SIAM Journal on Scientific Computing}, 35(3), C194-C212, 2013

\item Jack Poulson, Bryan Marker, Robert van de Geijn, Jeff R.\ Hammond, and 
Nichols Romero. Elemental: a new framework for distributed memory dense matrix 
computations, {\it ACM Transactions on Mathematical Software}, 39(2), 
13:1--13:24, 2013

\item Laurent Demanet, Matthew Ferrara, Nicholas Maxwell, Jack Poulson, and 
Lexing Ying. A butterfly algorithm for synthetic aperture radar imaging,
{\it SIAM Journal on Imaging Sciences}, 5(1):203--243, 2012

\item Bryan Marker, Ernie Chan, Jack Poulson, Robert van de Geijn, 
Rob F.\ Van der Wijngaart, Timothy G.\ Mattson, and Theodore E.\ Kubaska.
Programming many-core architectures - a case study: dense matrix computations on the Intel SCC processor, {\it Concurrency and Computation: Practice and 
Experience}, 24(12):1317--1333, 2012

\end{itemize}

\subsection*{Submitted Journal Articles}

\begin{itemize}

\item Martin D. Schatz, Robert A. van de Geijn, and Jack Poulson.
Parallel matrix multiplication: A systematic journey, {\it SIAM Journal on Scientific Computing}

%\item Jack Poulson and Robert van de Geijn.
%(Parallel) algorithms for two-sided triangular solves and matrix multiplication,
%{\it In limbo}

%\item Martin D. Schatz, Jack Poulson, and Robert A. van de Geijn. 
%Scalable Universal Matrix Multiplication Algorithms: 2D and 3D Variations on a 
%Theme, {\it In limbo}

\end{itemize}

%\subsection*{In Preparation}
%
%\begin{itemize}
%
%\item Jack Poulson and Lexing Ying.
%High-performance parallel sparse-direct triangular solves.
%
%\item Jack Poulson, Yingzhou Li, and Lexing Ying.
%Parallel $\mathcal{H}$-matrix algebra, Part I: Matrix-vector multiplication.
%
%\item Jack Poulson, Yingzhou Li, and Lexing Ying.
%Parallel $\mathcal{H}$-matrix algebra, Part II: Matrix-matrix operations.
%
%\end{itemize}

\subsection*{Conference Papers}

\begin{itemize}

\item Jack Poulson, Bj\"orn Engquist, Siwei Li, and Lexing Ying.
      A parallel sweeping preconditioner for frequency-domain seismic 
      wave propagation, {\it SEG 2012}

\item Bryan Marker, Jack Poulson, Don Batory, and Robert van de Geijn. 
      Designing linear algebra algorithms by transformation: mechanizing the 
      dense linear algebra expert, {\it iWAPT 2012}

\end{itemize}

%\subsection*{Dissertation}
%
%\begin{itemize}
%\item Fast parallel solution of heterogeneous 3D time-harmonic wave equations.
%      December 2012.
%\end{itemize}

\subsection*{Dissertation}
\begin{itemize}
\item Jack Poulson. Fast parallel solution of heterogeneous 3D time-harmonic wave equations. 
      Institute for Computational Engineering and Sciences, UT Austin. December 2012
\end{itemize}

\subsection*{Technical Reports}

\begin{itemize}

\item Martin Schatz, Jack Poulson, and Robert van de Geijn. 
      Parallel matrix multiplication: 2D and 3D, FLAME Working Note \#62. 
      The University of Texas at Austin, Department of Computer Science, 
      TR-12-13. June 2012

\item Bryan Marker, Andy Terrel, Jack Poulson, Don Batory, and 
      Robert van de Geijn. Mechanizing the dense linear algebra developer, 
      FLAME Working Note \#58. 
      The University of Texas at Austin, Department of Computer Science, 
      TR-11-18. April 2011

\item Jack Poulson, Robert van de Geijn, and Jeffrey Bennighof. 
      Parallel algorithms for reducing the generalized Hermitian-definite 
      eigenvalue problem, FLAME Working Note \#56.
      The University of Texas at Austin, Department of Computer Science,
      TR-11-05. February 2011

\item Jack Poulson, Bryan Marker, Jeff R. Hammond, Nichols A. Romero, and 
      Robert van de Geijn. Elemental: a new framework for distributed memory 
      dense matrix computations, FLAME Working Note \#44.
      The University of Texas at Austin, Department of Computer Science,
      TR-10-20. January 2011

\end{itemize}

%\subsection*{In Preparation}
%
%\begin{itemize}
%
%\item Jack Poulson, Lexing Ying, Nicholas Maxwell, and Laurent Demanet.
%A parallel butterfly algorithm for Fourier Integral Operators.
%
%\item Jack Poulson and Lexing Ying.
%Communication-optimal $\mathcal{H}$-matrix multiplication.
%
%\end{itemize}

%\subsection*{Master's Thesis}
%
%\begin{itemize}
%\item Formalized parallel dense linear algebra and its application to the 
%generalized eigenvalue problem. 2009, {\it The University of Texas at Austin}.
%Supervised by Jeffrey Bennighof.
%\end{itemize}

\section*{Invited Talks}

\begin{tabular}{rlll}
07/2015 & Computing Sciences Seminar      & LBNL                          & Berkeley, CA \\
07/2015 & MOPTA                           & Lehigh University             & Bethlehem, PA \\
06/2015 & JuliaCon                        & MIT                           & Cambridge, MA \\
05/2015 & Data Science Seminar            & Uber                          & San Francisco, CA \\
05/2015 & Berkeley-Inria-Stanford         & UC Berkeley                   & Berkeley, CA \\ 
04/2015 & Applied Math. Colloquium        & UCLA                          & Los Angeles, CA \\
03/2015 & Comput. Sci \& Eng.             & SIAM                          & Salt Lake City, UT \\
02/2015 & Lin. Alg. \& Opt. Seminar       & Stanford University           & Stanford, CA \\
12/2014 & Parallel Computing Lab          & Intel                         & Sunnyvale, CA \\
11/2014 & CASC Seminar                    & LLNL                          & Livermore, CA \\
11/2014 & Applied Math Seminar            & Stanford University           & Stanford, CA \\
10/2014 & AMCS Seminar                    & KAUST                         & Thuwal, KSA \\
10/2014 & ICL Seminar                     & University of Tennessee       & Knoxville, TN \\
10/2014 & Scientific \& Stat.\ Seminar    & University of Chicago         & Chicago, IL \\
09/2014 & BLIS Retreat                    & University of Texas at Austin & Austin, TX \\
09/2014 & SIAM-CCE Seminar                & MIT                           & Cambridge, MA \\
09/2014 & Applied Math Seminar            & MIT                           & Cambridge, MA \\
07/2014 & Annual Meeting                  & SIAM                          & Chicago, IL \\
07/2014 & Parallel Matrix Alg.\ and Appl. & USI                           & Lugano, Switzerland \\
01/2014 & Math+X Seminar                  & Stanford University           & Stanford, CA \\
12/2013 & Fall Meeting                    & AGU                           & San Francisco, CA \\
11/2013 & CSE Seminar                     & Georgia Institute of Tech.\   & Atlanta, GA \\
10/2013 & ICME LA/Opt Seminar             & Stanford University           & Stanford, CA \\
08/2013 & SEP Seminar                     & Stanford University           & Stanford, CA \\
06/2013 & ICS                             & ACM/SIGARCH                   & Eugene, OR \\
03/2013 & TCCS                            & University of Texas at Austin & Houston, TX \\
01/2013 & CSE Seminar                     & Georgia Institute of Tech.\   &  Atlanta, GA \\
12/2012 & Seminar                         & LBNL                          &  Berkeley, CA \\
11/2012 & TCCS                            & University of Texas at Austin & Austin, TX \\
11/2012 & Annual Meeting                  & SEG                           & Las Vegas, NV \\
06/2012 & Domain Decomposition 21         & Springer                      & Rennes, France \\
06/2012 & Applied Linear Algebra          & SIAM                          & Valencia, Spain \\
05/2012 & IAMCS                           & KAUST                         & Thuwal, KSA \\
04/2012 & SHAXc                           & KAUST                         & Thuwal, KSA \\
03/2012 & TCCS                            & University of Texas at Austin & Houston, TX \\
02/2012 & Parallel Processing             & SIAM                          & Savannah, GA \\
01/2012 & ICERM-TW12                      & Brown University              & Providence, RI \\
10/2011 & TCCS                            & University of Texas at Austin & Austin, TX \\
03/2011 & TCCS                            & University of Texas at Austin & Houston, TX \\
02/2011 & Comput.\ Sci.\ \& Eng.\         & SIAM                          & Reno, NV \\
02/2010 & Parallel Processing             & SIAM                          & Seattle, WA \\
\end{tabular}

\section*{Teaching}
\begin{tabular}{lll}
Fall 2015 & Differential Equations with Linear Algebra & Stanford University \\
          &      & Undergraduate-level \\
  & & \\
Spring 2015 & Mathematics in the Real World               & Stanford University \\
            & \href{http://web.stanford.edu/~poulson/courses/MitRW15}{\tt web.stanford.edu/\textasciitilde poulson/courses/MitRW15} & Undergraduate-level (non-major) \\
 & & \\
Winter 2015 & Advanced Topics in Numerical Linear Algebra & Stanford University \\
            & \href{http://web.stanford.edu/~poulson/courses/ATiNLA15}{\tt web.stanford.edu/\textasciitilde poulson/courses/ATiNLA15} & Graduate-level \\
 & & \\
Spring 2014 & Fast Linear Algebra                         & Georgia Institute of Tech.\ \\
            & \href{http://web.stanford.edu/~poulson/courses/FLA14}{\tt web.stanford.edu/\textasciitilde poulson/courses/FLA14} & Graduate-level \\
 & & \\
Spring 2013 & Introduction to MPI (short course)          & Stanford University \\
            & & Graduate-level \\
\end{tabular}

%\section*{Mentoring}
%\begin{itemize}
%\item Austin Benson, Stanford University, PhD\ student in Computational and Mathematical Engineering
%\item Yingzhou Li, Stanford University, PhD\ student in Computational and Mathematical Engineering
%\end{itemize}

\section*{Academic Service}
%\subsection*{Journals}

\subsection*{Editing and Review}
\begin{tabular}{ll}
Reviewer & ACM Transactions on Mathematical Software \\
Reviewer & Advances in Applied Mathematics \\
Reviewer & Computing \\
Reviewer & IEEE Antennas and Wireless Propagation Letters \\
Reviewer & IEEE Transactions on Signal Processing \\
Reviewer & International Journal of High Performance Computing \\
Reviewer & Journal of Computational Physics \\
Reviewer & Journal of Parallel and Distributed Computing \\
Reviewer & PeerJ Computer Science \\
Reviewer & SIAM Journal on Computing \\
Reviewer & SIAM Journal on Scientific Computing \\
External Reviewer & Communications in Mathematical Sciences \\
External Reviewer & SIAM Journal on Numerical Analysis \\
Guest Editor & ACM XRDS, 2012
\end{tabular}
%\begin{itemize}
%\item Reviewer, ACM Transactions on Mathematical Software
%\item Reviewer, Advances in Applied Mathematics
%\item Reviewer, Computing
%\item Reviewer, IEEE Antennas and Wireless Propagation Letters
%\item Reviewer, IEEE Transactions on Signal Processing
%\item Reviewer, International Journal of High Performance Computing
%\item Reviewer, Journal of Parallel and Distributed Computing
%\item Reviewer, SIAM Journal on Computing
%\item Reviewer, SIAM Journal on Scientific Computing
%\item External Reviewer, Communications in Mathematical Sciences
%\item External Reviewer, SIAM Journal on Numerical Analysis
%\item Guest Editor, ACM XRDS, 2012
%\end{itemize}

\subsection*{Conference Stewardship}
\begin{tabular}{lll}
Program Committee Member & IPDPS & 2015-2016 \\
Editorial Board Member & SISC Special Section for CSE15 (CSE Software) & 2015 \\
Program Committee Member & IEEE CSE & 2015 \\
Program Committee Member & XSEDE & 2014 \\
Program Committee Member & PGAS & 2014 \\
Program Committee Member & IMUDI & 2012-2013 \\
%Member    & NSF Review Panel & 2014
\end{tabular}
%\begin{itemize}
%\item PC member, IPDPS 2015
%\item PC member, XSEDE14
%\item PC member, PGAS14
%\item Member, NSF Review Panel, 2014
%\item PC member, IMUDI 2012-2013
%\end{itemize}

\subsection*{Departmental Service}
\begin{tabular}{llll}
Chair            & PhD Committee for Dr. Yuekai Sun & Stanford University & 2015 \\
Chair            & PhD Committee for Dr. Rustem Zaydullin & Stanford University & 2015 \\
Area Coordinator & CSE/CS-CSE MS program & Georgia Tech & 2014 \\
Member           & CSE/CS-CSE PhD Admissions Committee  & Georgia Tech & 2014 \\
Member           & PhD Committee for Dr. Xiaolin Wang   & Georgia Tech & 2014 \\
Member           & PhD Committee for Dr. Christopher Turnes & Georgia Tech & 2014 
\end{tabular}
%\begin{itemize}
%\item Area Coordinator, Georgia Tech CSE/CS-CSE MS program, 2014
%\item Member, Admissions Committee for Georgia Tech CSE/CS-CSE PhD program, 2014
%\item Member, PhD committee for Xiaolin Wang (Georgia Tech,Math)
%\item Member, PhD committee for Christopher Turnes (Georgia Tech,ECE)
%\end{itemize}

%\section*{Professional Societies}
%\begin{itemize}
%\item Member, Association for Computing Machinery (ACM)
%\item Member, American Geophysical Union (AGU)
%\item Member, Society for Industrial and Applied Mathematics (SIAM)
%\end{itemize}

\section*{Funding and Awards}
\begin{itemize}
\item 2015 - DARPA XDATA, AFRL Contract FA8750-12-2-0306
\item 2014 - DARPA XDATA, AFRL Contract FA8750-13-C-0002
\item 2013 - NSF SSI-SI2 Supplement, Co-PI, ACI-$1148125$
\item 2013 - UT Austin Outstanding Dissertation Award
\item 2009 - Computational and Applied Mathematics Fellowship
\end{itemize}

%\section*{Employability}
%\begin{itemize}
%\item United States Citizen.
%\end{itemize}

\bigskip

% Footer
\begin{center}
  \begin{footnotesize}
    Last updated: \today \\
    \href{\footerlink}{\texttt{\footerlink}}
  \end{footnotesize}
\end{center}

\end{document}
